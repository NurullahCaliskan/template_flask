\documentclass[10pt,a4paper]{book}
\usepackage[turkish]{babel}
\usepackage[utf8]{inputenc}

\usepackage{makeidx}
\usepackage{graphicx}
\makeindex
\author{Ilker Manap}
\title{Flask Taslak Kullanımı}

\begin{document}
\maketitle
\tableofcontents
\chapter{Giriş}

Flask ile yazılmış uygulamaların bir web sunucu üzerinden sunulması çoğu kişi için 
başa çıkılamaz karmaşık işlemler gerektiren bir süreç gibidir. Birden fazla konuda 
doğru ayarlamalar yapmak gerektiği için, konuyu az bilenler tarafından yapılmaya 
çalışıldığında sorun çıkması ihtimali de yüksektir.

Bu belge ile, karmaşık görünen işlemlerin daha kolay yapılabilmesini sağla-maya 
çalışacağız. 

Uygulamamızın, systemd kullanan bir sunucuda, sunucunun servisi olarak çalışmasını 
sağlayacağız. 


\chapter{Sanal Domain Nasıl Ayarlanır?}
DNS\index{DNS} ve nginx\index{nginx} ayarları
\section{DNS}
Dns detayları

\section{Nginx Ayarları}
Nginx

\chapter{Flask Kurulumu}



\chapter{Flask Uygulamanın Sunucuya Kurulması}

\printindex
\end{document}